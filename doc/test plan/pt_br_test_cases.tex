%
% Portuguese-BR vertion
% 
\documentclass{article}

\usepackage{ipprocess}
% Use longtable if you want big tables to split over multiple pages.
% \usepackage{longtable}
\usepackage[utf8]{inputenc} 
\usepackage[brazil]{babel} % Uncomment for portuguese

\sloppy

\graphicspath{{./pictures/}} % Pictures dir
\makeindex
\begin{document}

\DocumentTitle{Plano de Testes}
\Project{Core-MUSA}
\Organization{Universidade Estadual de Feira de Santana}
\Version{Build 2.0a}

\capa
\newpage

%%%%%%%%%%%%%%%%%%%%%%%%%%%%%%%%%%%%%%%%%%%%%%%%%%
%% Revision History
%%%%%%%%%%%%%%%%%%%%%%%%%%%%%%%%%%%%%%%%%%%%%%%%%%
\section*{\center Histórico de Revisões}
  \vspace*{1cm}
  \begin{table}[ht]
    \centering
    \begin{tabular}[pos]{|m{2cm} | m{7.2cm} | m{3.8cm}|} 
      \hline
      \cellcolor[gray]{0.9}
      \textbf{Date} & \cellcolor[gray]{0.9}\textbf{Descrição} & \cellcolor[gray]{0.9}\textbf{Autor(s)}\\ \hline
      \hline
      \small 10/11/2014 & \small Concepção do Documento & \small fmbboaventura \\ \hline      
       \small 16/11/2014 & \small  Plano de Teste da Memória de Instrução; & \small mtcastro \\ \hline     
         \small 16/11/2014 & \small  Correções na Introdução, Refatoração do Plano para ULA, Plano de Testes do Banco de Registradores; & \small Odivio Caio \\ \hline   
   
    \end{tabular}
  \end{table}

\newpage

% TOC instantiation
\tableofcontents
\newpage

%%%%%%%%%%%%%%%%%%%%%%%%%%%%%%%%%%%%%%%%%%%%%%%%%%
%% Document main content
%%%%%%%%%%%%%%%%%%%%%%%%%%%%%%%%%%%%%%%%%%%%%%%%%%
\section{Introdução}

  \subsection{Objetivo}
  Este documento tem como objetivo descrever os casos de teste dos componentes do processador de propósito geral MUSA. Serão apresentados os conjuntos de condições ultilizados para cada caso de teste, suas entradas, ações, saidas e o fluxo de operação.
  
  \subsection{Visão Geral do Documento}
  \begin{itemize}
    \item Sessão 2: Casos de Teste: Lista dos casos de testes do projeto Core-MUSA.
    % \item Referências: provê uma lista completa de todos os artefatos referenciados nesse documento.
  \end{itemize}
  
  \subsection{Definições, Acrônimos e Abreviações}
  \FloatBarrier
    \begin{table}[H] 
      \begin{center}
        \begin{tabular}[pos]{|m{2cm} | m{8cm}|} 
          \hline 
          \cellcolor[gray]{0.9}\textbf{Termo} & \cellcolor[gray]{0.9}\textbf{Descrição} \\ \hline
          TC & Caso de Teste  \\ \hline
          SB & Sub-fluxo \\ \hline
          FS & Fluxo Secundário \\ \hline
          NFR & Requisito Não Funcional \\ \hline
          FR & Requisito Funcional \\
          \hline
        \end{tabular}
      \end{center}
    \label{tab:definicoes}
    \end{table}
  
  \section{Casos de Teste}
  Esta sessão apresenta o conjunto de TC realizados para a implementação dos testes do projeto Core-MUSA. As sessões a seguir foram divididas e nomeadas utilizando a nomenclatura abreviada [TC (NÚMERO DO TC)] seguido de uma breve descrição em forma de título.

  \testcase{ULA}
A ULA tem como objetivo principal realizar operações logicas e aritimeticas, onde algumas delas estão ligadas diretamente a flags informativas ou de erros.
  
  \inputs
  	\begin{itemize}
     \item Operando 1;
     \item Operando 2;
     \item Sinal de identificação da operação;
     \end{itemize}
    
  \actions
  \begin{itemize}
     \item Realizar a operação solicitada;
     \item Ativar os sinais de saida de dados e de flags, caso ocorram;
    \end{itemize}
  
  \results
  	\begin{itemize}
     \item Valor de 32 bits relativos ao resultado da operação;
     \item Sinal de flag, caso ocorram;
    \end{itemize}
  
  % descricao do fluxo principal de eventos
  \begin{mainflow}
    \item As funcionalidades serão testadas na seguinte ordem de acordo com o sinal de identificação: ADD, ADDI, SUB, SUBI,
AND, ANDI, OR, ORI, MUL, DIV, CMP, NOT;
    \item Os valores ultilizados para os operandos serão escolhidos de forma aleatória;
    \item Cada funcionalidade será testada 100 vezes;
    \item Serão testados os seguintes casos de flags auxiliares: Equals ,Above;
    \item Serão testados os seguintes casos de flag de erro de forma proposital: Overflow;
    \item As flags serão testadas com a ultilização de uma saida de controle temporária para os sinais;
  \end{mainflow}
  
  % descricao do fluxo secundário (quando existir)
    %\begin{secondaryflow} 
    %  \sfitem{Título do Fluxo Secundário}
    %  \begin{enumerate}
    %    \item Liste aqui as etapas do fluxo secundário;
    %  \end{enumerate}
    %  \sfitem{Título do Fluxo Secundário}
    %  \begin{enumerate}
    %     \item Liste aqui as etapas do fluxo secundário;
    %  \end{enumerate}
    %\end{secondaryflow}  
  
  
  
  \testcase{Memória de Instrução}
O objetivo deste teste é garantir que os registradores responsáveis por armazenar as instruções do programa  estejam lendo as informações corretas na posição correta.
  
  \inputs
  	\begin{itemize}
     \item Endereço de Instrução;
     \end{itemize}
    
  \actions
  \begin{itemize}
     \item Busca um instrução na posição informada no endereço;
    \end{itemize}
  
  \results
  	\begin{itemize}
     \item Confirmação da veracidade dos dados informado na saída;
    \end{itemize}
  
  % descricao do fluxo principal de eventos
  \begin{mainflow}
    \item A memória de instrução recebe um endereço, a instrução
Relacionada é buscada e passada adiante;
    \item A entrada será um código de endereço aleatório  de 18 bits;
    
    \item Será testada 20 vezes;
    \item A condição de testes é que o valor contido nos registradores de instrução seja igual ao valor associado ao endereço de instrução do programa de teste;
    
    \item Critério de aceitação é acertos igual ou maior 99\% do total de casos de teste;
    \item As flags serão testadas com a ultilização de uma saida de controle temporária para os sinais;
  \end{mainflow}

 \testcase{Bando de Registradores}
O Banco de Registradores tem como função executar operações de 
leitura de dados anteriormente gravados e de escrita de dados para modificar as informações internas. Ele possui 32 registradores de propósito geral do processador. 
  
  \inputs
  	\begin{itemize}
     \item Identificador(es) do(s) Registrador(es) de Leitura;
     \item Sinal de ativação de Leitura;
     \item Identificador do Registrador de Escrita;
     \item Sinal de ativação de Escrita;
     \item Valor que será guardado no Banco;
     \end{itemize}
    
  \actions
  \begin{itemize}
     \item Realizar a Leitura do(s) Registrador(es) solicitado(s);
     \item Realizar a escrita do valor solicitado no registrador objetivo;
    \end{itemize}
  
  \results
  	\begin{itemize}
     \item Valor(es) de 32 bits relativos ao resultado da(s) Leituras(s); 
    \end{itemize}
  
  % descricao do fluxo principal de eventos
  \begin{mainflow}
     \item Serão escritos valores escolhidos pela equipe em 10 registradores do Banco.
     \item Será feita a Leitura do 10 registradores escolhidos previamente no Banco.
     \item Serão escritos valores escolhidos pela equipe em 5 dos 10 registradores previamente escolhidos.
     \item Será feita novamente a leitura dos 10 registradores escolhidos previamente no Banco.
  \end{mainflow}

 \testcase{Pilha}
A pilha é tem como funcão armazenar os valores de PC em casos de Chamada a função. Possui 32 registradores de 18 bits e um contador responsável por apontar o topo da pilha.
  
  \inputs
  	\begin{itemize}
     \item Sinal de ativação de Leitura;
     \item Saida do endereço de 18bits;
     \item Sinal de ativação de Escrita;
     \item Entrada do endereço de 18bits.
     \end{itemize}
    
  \actions
  \begin{itemize}
     \item Trocar o valor de saida para o proximo na pilha;
     \item Modificar o contador para apontar sempre para o topo da pilha;
    \end{itemize}
  
  \results
  	\begin{itemize}
     \item Valor de 18 bits relativo ao ultimo valor guardado; 
    \end{itemize}
  
  % descricao do fluxo principal de eventos
  \begin{mainflow}
     \item Serão escritos valores aleatorios de  18bits na pilha.
     \item A pilha será esvaziada e os valores comparados.
  \end{mainflow}

% Optional bibliography section
% To use bibliograpy, first provide the ipprocess.bib file on the root folder.
% \bibliographystyle{ieeetr}
% \bibliography{ipprocess}

\end{document}
